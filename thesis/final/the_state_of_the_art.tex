\chapter{The State of The Art}
\labelchapter{the_state_of_the_art}
%-----------------------
\beginchapter
%-----------------------
\begin{comment}
=======================================

[there are some stuff that i wrote during the research, i will try to re-use them, it's still not complete]

say that i investigated a number of tools, which type of tools, what was the investigation criteria, state the features that of the comparision criteria, meaning of $+ and -$ and put the table.

each feature, take each tool with + or ++ divide them into categories, compare the tools from the point of view of each feature together.

add conclusion @ the end\\
========================================
\end{comment}

Before the start of the implemetation, a review of currently existing diagramming tools was done regarding the features they provide as well as the future planned ones. This review was necessary to find out the features related to keyboard navigation and editing in current diagramming tools. Also previous research done in that field was considered.

\section{Currently Available Tools}
The tools investigated fall under many categories, UML tools, general purpose diagramming tools, and so on. In total NUMBER OF TOOLS were investigated. The investigation checked if the tools contain any of the six features listed below, how well is it developped, its usability and efficiency. Only the tools containing relevat features were considered. The relevant tools were organized in a table REFERENCE TABLE and compared based on the presence of one or more of the six features subject of investigation.

\begin{itemize}
\item {\bf Feature 1:}
\par \noindent
Smart Navigation considering the position of the nodes and its connections. This feature is more advanced than simple tab-based navigation.

\item {\bf Feature 2:}
\par \noindent
Deleting shapes using keyboard.

\item {\bf Feature 3:}
\par \noindent
Dragging shapes using keyboard.

\item {\bf Feature 4:}
\par \noindent
Editing shape properties using keyboard.

\item {\bf Feature 5:}
\par \noindent
Adding shapes using keyboard.

\item {\bf Feature 6:}
\par \noindent
Creating diagram using a script.
\end{itemize}

\begin{center}
	%\begin{table}
	\begin{tabular}{ | p{3.5cm} | l | l | l | l | l | l |}
	\hline
	{\bf Tool} & {\bf Feat. 1} & {\bf Feat. 2} & {\bf Feat. 3} & {\bf Feat. 4} & {\bf Feat. 5} & {\bf Feat. 6}\\ \hline \hline 
	Papyrus UML & + & ++ & - & - & - & -\\ \hline
	ArgoUML & - & ++ & ++ & - & - & -\\ \hline
	Violet UML Editor & - & ++ & - & - & - & -\\ \hline
	yEd & ++ & ++ & - & - & - & -\\ \hline
	OpenOffice Draw & - & ++ & ++ & + & - & -\\ \hline
	xfig & - & - & - & - & -* & -\\ \hline
	%IBM Webshpere Studio & + & ++ & ++ & ++ & ++ & -\\ \hline
	IBM Rational Application Developer & ++ & ++ & ++ & - & - & -\\
	\hline
	\end{tabular}
	%\caption{Comparision of Tools}
	%\label{table:comparision_of_tools)}
	%\end{table}
\end{center}


\section{Previous Research}
