\chapter*{Abstract}
\labelchapter{abstract}
%-----------------------
\beginchapter
%-----------------------
\begin{comment}
The abstract is a very important and obligatory part of the report
allowing the potential reader to judge, if he is the target of this
report.  The abstract should be written with broadly understandable
technical language and should be self-contained, i.e. should not
contain any references or citations. Also the usage of abbreviations
and acronyms should be avoided, since the same acronyms or
abbreviations can have different meanings on different research
fields.  The abstract should typically contain about 300 words and in
any cases not more than one page.  The definition of the research
field and the most important outcome of the presented research are the
obligatory components of the abstract.
\end{comment}

\vspace{2cm}

\begin{spacing}{1.5}
Most diagramming tools rely heavily on drag and drop actions done using pointing devices. Very few tools support some accessibility tools to help the people that cannot handle pointing devices effectively, because of blindness or other disability, or simple people who don't have one. But yet, still this support was not yet fully developped and integrated in a usable way. A well designed keyboard support for creating, editing and navigating diagrams needed to be designed. In this thesis, previously implemented features in many diagramming tools were investigated and compared. The integrity of the concepts as well as how to contruct a complete and efficient diagram navigation and editing using the keyboard were discussed. New as well as already existing concepts that help creating, editing and navigating diagrams were organized and explained. Some of the concepts were implemented a prototype as an example for the discussed ideas. The constructed prototype can be used to create diagrams using the keyboard, navigate them, and edit the nodes' properties. This work paves the way to integrate such support in commercial diagramming tools. This integration will offer to the blind and disabled users a easy way to create diagrams as professinally as any other sighted or non-disabled person.
\end{spacing}