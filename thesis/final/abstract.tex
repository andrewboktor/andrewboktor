\chapter*{Abstract}
\labelchapter{abstract}
%-----------------------
\beginchapter
%-----------------------

\vspace{1.5cm}

\begin{spacing}{1.5}
Most diagramming tools rely heavily on drag and drop actions done using pointing devices. Very few tools support some accessibility tools to help the people that cannot handle pointing devices efficiently, because of blindness or other disability, or simply people who don't have one. But yet, still this support was not yet fully developed and integrated in a usable way. A well designed keyboard support for creating, editing and navigating diagrams needed to be designed. Such support would also be very useful for professionals to enhance their productiveness through removing the need of using drag and drop actions and replacing them with keyboard actions. In this thesis, previously implemented features in many diagramming tools were investigated and compared. The integrity of the concepts as well as how to construct a complete and efficient diagram navigation and editing using the keyboard were discussed. New as well as already existing concepts that help creating, editing and navigating diagrams were organized and explained. Some of the concepts were implemented in a prototype as an example for the discussed ideas. The constructed prototype can be used to create diagrams using the keyboard, navigate them, and edit the nodes' properties. This work paves the way to integrate such support in commercial diagramming tools. This integration will offer to the blind and disabled users an easy way to create diagrams as professionally as any other sighted or non-disabled person can and offer to the professionals an efficient and usable tool that emphasizes productiveness.
\end{spacing}
