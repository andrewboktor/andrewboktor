\chapter{Conclusion}
\labelchapter{conclusion}
%-----------------------
\beginchapter
%-----------------------

\section{Conclusion}
Currently, most of the existing diagramming tools support only navigation of diagrams using pointing devices. While the use of pointing devices might seem for some people the easiest and most efficient way for diagram navigation, for others, due to visual impairments or motoric disabilities, the use of such devices is very hard if not impossible. This prevents such people from using general purpose diagramming tools to express their ideas and/or do professional work using them. Furthermore, professionals also are usually less productive when they need to switch often between the keyboard for textual input and a pointing device for diagram editing. 

\paragraph{}
To enable visually impaired people, and people with motoric disabilities to use diagramming tools effectively, an efficient method had to be developed to the whole diagram creating and editing process without any poiting devices. This thesis contributes to the development of such a method in the following means:
\begin{itemize}
\item {\bf Doing a research} that investigates the currently existing diagramming tools, their future planned features, and their accessibility tools giving a result about what is currently implemented and available in the market.
\item {\bf Defining and Comparing concepts} related to the matter of efficient creation, editing, and navigation of diagrams using only the keyboard.
\item {\bf Implementing} some of these concepts and providing example code to be ready for integration in commercial diagramming tools.
\end{itemize}

\paragraph{}
A thorough investigation of the currently existing diagramming tools was conducted. The tools investigated were organized and checked if they have relevant features to the topic of this thesis or not. Most of the tools with relevant features were downloaded and installed, the relevant features were described in details regarding how they can be used and regarding their efficiency. A table was constructed to show the summary of the comparison of the tools. The details of investigated tools and their comparison can be found in the State of The Art Chapter (REFERENCE)

\paragraph{}
Throughout the work on this thesis, many concepts were defined and compared. A summary with the concepts discussed is displayed below. The concepts compared mainly reside in two groups, ``Keyboard Navigation of Diagrams'' and ``Keyboard Creation and Editing of Diagrams''. Many strategies and implementation approaches were discussed under each topic. The details of the comparison and discussion of those concepts can be found in the Concepts Chapter (RFERENCE)

\begin{spacing}{2}\end{spacing}
\par \noindent
{\bf Summary of discussed concepts:}
\begin{itemize}
\item {\bf Keyboard Navigation}

\begin{itemize}
	\item {\bf Tab Navigation}
	\par \noindent
	Tab Navigation is the easiest to implement but hardly usable.

	\item {\bf Arrow Keys Navigation}
	\par \noindent
	Arrow keys Navigation is very efficient to navigate diagrams, and very easy to use if implemented in {\em Approach 3} or {\em Approach 4} discussed in the {\em Concepts}\refchapter{concepts}.

	\item {\bf Smart Navigation Following Connections}
	\par \noindent
	Smart Navigation considers the semantics of the connections while navigating, it can also change its behavior based on the meaning of the connections, e.g. in directed graphs. It can be used combined with {\em Arrow Keys Navigation} to provide compelete navigation of the diagram.

	\item {\bf Diagram Navigation Completeness}
	\par \noindent
	For the diagram navigation to be complete, i.e. the user can reach every node/shape in the diagram, one or more of the navigation methods might have to be implemented in conjunction.
\end{itemize}

\item {\bf Diagram Editing Using the Keyboard}
\begin{itemize}
	\item {\bf Editing Using Accelerators} 
	\par \noindent
	Editing Using Accelerators is useful in some cases, but in most cases it is hard for the user to remember all the accelerators and the functionality for each one.
	
	\item {\bf Menu Based Editing}
	\par \noindent
	Menu based editing is the best way for editing a diagram, it provides easy access to all the object properties in a way that doesn't require the user to remember a lot of commands or shortcuts.
	
	\item {\bf Command Based Editing}
	\par \noindent
	Command Based Editing is useful to emphasize productiveness or as a part of the scripting. Depending on the application and the type of users, sometimes command based editing is more usable than the {\em Menu Based Editing}.
	
	\item {\bf Scripting}
	\par \noindent
	Using Scripting, one can automate the diagram creation process through inputing a script to the diagramming tool and the diagramming tool will automatically generate the diagram.
\end{itemize}

\end{itemize}

\paragraph{}
After the definition and comparison of the concepts, some of them were chosen to be implemented. Those ones that were implemented and the details of their implementation are described in the Implementation Chapter (REFERENCE). During the implementation, a prototype demostrating how many of the functions described can be integrated together in one tool and how each can be implemented and used in an efficient way. The prototype includes the {\em Command Based Editing} functionality for which a small grammar was defined and used. It includes also many types of navigation like {\em Tab Navigation}, {\em Arrow Keys Navigation} and {\em Double Stepped Smart Navigation} all integrate to work in cooperation with each other and with the commands. Based on the implementation, some improvements for the implemented features were made, and other improvements were suggested as future work that can be done on this topic.


\section{Recommendations for Future Work}
Due to the short time and the big amount of work, the implemented features were limited. Thus, some recommendation for the future work and/or research are stated below. These recommendations for future work are aiming at enriching the persons willing to work on that topic with ideas about how to improve the existing or develop new features related to this topic.

\begin{itemize}
\item The actions triggered by the keyboard in order to achieve the needed navigation or editing of the diagram can be mapped to voice commands in order to facilitate them for the disable people. As voice commands require no special movement of the hands or realization of exact distances as it is the case when using the keyboard.

\item According to the Scripting language described in the implementation chapter (REFERENCE), it can be extended to support drawing starting from a reference point, not only using absolute values for the position of the objects.

\item The scripting language can also be extended to support the scaling of the canvas or the change of the coordinates before the insertion of a script file, or any other chunk of code, which will allow the drawing of scaled, inverted, or shifted diagrams.

\item Implementing different aspects of diagram navigation and keyboard editing and conducting a user evaluation on the implemented prototype will be very beneficial to define which of the different aspects is more usable and behaves naturally from the point of view of the user.

\item The integration of all the previously mentioned implementations into a commercial diagramming tool will get these functionalities out of the research field to the market and thus allowing the target users to benefit from them.
\end{itemize}
