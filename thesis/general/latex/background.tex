\chapter{Background}\label{chap:background} 


In this chapter, you should present the background information needed for the
proposed work of the thesis. 


The background chapter is typically subdivided in several subsections.
In some cases it is also more suitable to make more than one main
chapter with very diverse theoretical backgrounds, if needed.
The background chapter (or chapters) should not constitute
more than 25-30\% of the total text.


It is recommendable that each chapter begins by an introductory 
paragraph or two. It is like a mini-introduction to this chapter. 


\section{First Section}

Especially important in the background  presentation are the references on
the most acknowledged books, articles or crucial WWW-links.  The most
efficient way to do it, is by use of the special tool called BibTeX.
It is a simple date base with formalized entries for the referred
literature.  An example of such entries you can find in the document
''bachelor.bib''.  By running the BibTeX in the LaTeX Command
Prompt window you can link from the \verb+.bib+ document only this entries,
which are actually cited in your report.  This allows to use the same
literature data base for different documents.  Your literature list
attached at the end of the document will be either in the sequence of
referring in the text or alphabetic.

\subsection{Getting BibTeX working}
\label{ss:bib-attach}

To get the bibliography attached:

\begin{enumerate}
    \item prepare your reference data base document (here: bachelor.bib)
    \item place the \verb"bibligraphystyle" command in backmatter.
    \item place the \verb"bibliography" command in backmatter.
    \item run LaTeX
    \item run \verb"Bibtex bachelor" in Command Prompt
    \item run LaTeX twice
\end{enumerate}

\subsection{Examples of references}
\label{ss:bib-ref-exe}


This is for example the reference on a very common textbook \cite{AbdennadherFruhwirth02},
and this is a reference on very interesting journal article \cite{AbdennadherRigotti05}.




\section{Another Section}
