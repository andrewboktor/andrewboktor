%%%%%%%%%%%%%%%%%%%%%%%%%%%%%%%%%%%%%%%%%%%%%%%%%%%%%%%%%%%%%%%%%%%%%
\chapter{Introduction}
%Give a label so that you can refer to this chapter:
\label{chap1}


The main task of the introduction in the report is to present the
logical chain

\begin{itemize}

\item starting from the current state-of-art,

\item evaluation of its weaknesses,

\item formulation of the main problem,

\item showing the way of solving the problem and

\item finishing with the contribution of the presented work to this solution.

\end{itemize}


The introduction can consist of a one text body, but it can also be
differently structured. Typical components are the motivation and
the aim of the project. The introduction has typically 1 to 4 pages.


\section{Motivation}
\label{s:Motivation}

The motivation should answer following questions:
Why the work has been done? For whom? What is its importance?


\section{Aim of the project}
\label{s:aim}

The description of the project task should describe
what is the technical aim of the work? what are the specific
tasks? wow it can be proved, that the task is reached?  \\
